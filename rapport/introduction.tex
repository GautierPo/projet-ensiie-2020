\section{introduction}
    Lors de conventions basées sur la culture japonaise(japanExpo, bakanim), il n'est pas rare d'y trouver des séances de Karaoké, ou tout un public se regroupe et chante au rythme des plus grands chefs d'oeuvre nippons.\newline
Cependant, si les visiteurs viennent à ces séances, c'est qu'ils s'attendent à chanter des musiques qu'ils connaissent et aiment. Pourtant les playlists sont souvent préfaites, et les "chanteurs" n'ont pas la possibilité de choisir sur quel hymne ils vont épuiser leurs cordes vocales. \newline
Voici donc le site Bakaraoké: gestionnaire de playlist, ou des utilisateurs peuvent rajouter à leur guise les musiques qu'ils veulent chanter. \newline

Nous étions 4 pour ce projet, la répartition des tâches fut:
\begin{itemize}
	\item Châtelet Leo: implémentation des playlists
	\item Cecconi Quentin: style css et pages html
	\item Goyard Louis: implémentation de Lektor
	\item Sadler Alec: système d'utilisateurs et formulaires
\end{itemize}
\newline



