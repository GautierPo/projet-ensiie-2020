\section{Apparence visuel}
\subsection{Mod�le Vue/Controlleur}
Pour l'apparence visuel du site nous avons adopt� le mod�le vue/controlleur. \\
Cela s'av�re particuli�rement utile pour l'affichage du header qui permet � l'utilisateur d'acc�der � toutes les fonctionnalit�s du site, et ce quelque soit la page sur laquelle il se trouve actuellement.\\
Chaque page du site inclus donc la vue de ce header en ent�te, puis une vue de la page en question.
Par exemple la page login.php inclus la vue header.php ainsi que la vue de login.php (une autre page qui n'est pas dans public).\\
L'affichage des images de profil est aussi g�rer par des vues et plus pr�cis�ment par des fonctions php qui permettent d'afficher une image et de d�cider de ses dimensions ainsi que de sa forme en retournant une balise avec du CSS.\\


\subsection{CSS}
Pour assurer qu'aucun style directement impl�ment� dans le html ne se fasse �craser par la feuille de style, l'utilisation de classe � �t� privil�gi� pour mieux contr�ler les balises sur lesquelles le style s'applique. Une feuille de style, style.css, est commune � toutes les pages (car pr�sente dans le head que l'on inclus syst�matiquement). Il est cependant possible d'int�grer d'autres feuilles de style pour des pages sp�cifiques � chaque page.
