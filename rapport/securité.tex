\section{Sécurité}
    Un minimum de sécurité a été implanté pour éviter des attaques récurrentes comme un ddos ou des injections SQL :\newline
Chaque variable obtenue d'un formulaire via la méthode POST sera d'abord filtré avec la commande htmlspecialchars() pour éviter que des caractères spéciaux apparaissent. Ensuite, ces variables seront rentrées en paramètres dans les commandes SQL via la méthode bindParam du PDO.\newline
Lors d'une création, d'une connexion ou de la modification d'un compte, on vérifie bien que les champs remplis sont bien valides (email au bon format, pas d'espace dans le username), ainsi nous n'aurons pas de risque concernant l'utilisateur qui rentre une commande de type 'DROP TABLE'.\newline
Chaque formulaire aura avant tout une validation javaScript pour ne pas surcharger le serveur par de mauvaises requêtes.\newline\newline

Pour éviter que l'utilisateur ne "spam" les listes de lectures, on utilise le fichier "ddosPreventer.php" qui limite à 6 le nombre de requêtes toutes les 30 secondes. On utilise aussi ce fichier au moment du login ou de l'inscription. 

