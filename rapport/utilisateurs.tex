\section{Utilisateurs}

Lorsque qu'un client arrive pour la première fois sur le site, il n'aura pas accès aux différents services du site. Il devra d'abord se créer un compte sur la page registration.php. Ici il entrera ses 
identifiants comme son adresse email, son nom d'utilisateur, et son mot de passe.
Il est à noter qu'un utilisateur ne peut pas avoir d'espace dans son nom, ce qui permet d'avoir une sécurité supplémentaire contre les injections SQL.

Après vérification javascript d'un formulaire, on récupère et envoie ces informations à la page addUser.php qui va les insérer dans la table "user", le mot de passe sera hashé pour éviter toute tentative de vol de mot de passe. Tout transfert d'information se fera via la méthode \texttt{POST}, afin de transférer les données de manière sécurisée.

Ces informations peuvent être  ultérieurement modifiées sur la page \texttt{/modifyUser.php}, qui enverra les modifications à \texttt{/Forms/modifyUserAccount.php}.

Une fois qu'un utilisateur s'est créé un compte, il peut se connecter via login.php. Si il le fait, on démarre une session et on initialise les attributs de la variable de session, c'est avec ces attributs qu'on déterminera si la session est ouverte, qu'on vérifiera ses droits et qu'on affichera les cosmétiques de l'utilisateur.

\subsection{Personnaliser son compte}

Aussi, après ajout d'un utilisateur dans "user", on insère une nouvelle ligne dans la table "UserCosmetics", cette table permet de savoir quel sont les cosmétiques choisis par l'utilisateur.  
Chaque utilisateur aura accès à une liste d'images de profil et de titres qu'il pourra sélectionner pour customiser son profil. Ces cosmétiques sont contenus dans les tables "Image" et "Titles".

Ces images ne sont pas toutes disponibles dès le départ. En effet chaque utilisateur possède des points d'expérience, initialisés à 0 à la création du compte, qui peuvent être gagné lorsqu'il rajoute un karaoke à la queue. Les images se débloquent au fur et à mesure que l'utilisateur passe du temps sur le site. Sur \texttt{ChangePP.php} seules les images et titres dont \texttt{xpNeeded<\$\_SESSION\lbrack'xp'\rbrack} sont affichés.\newline

Pour pouvoir modifier son compte, l'utilisateur doit se rendre sur changePP.php, pour modifier son image de profil et/ou son titre. Un formulaire envoie les informations à modifyUserCosmetics.php. L'utilisateur a le choix entre les différentes images contenues dans la table "Images" et les différents titres dans "Titles". \newline

\subsection{Les playlists}

Un utilisateur peut créer ses propres playlists de karas disponibles dans la table \texttt{karas}.
D'ici il pourra ajouter cette playlist dans la queue du lecteur (à condition d'être administrateur), ou simplement retrouver les karas qu'il préfère afin de les ajouter.
Toutes ces informations sont contenues dans la table \texttt{playlist}, et les utilisateurs peuvent regarder les playlists des autres utilisateurs si elles sont paramétrées comme étant publique.

Une playlist ne peut cependant être éditée que par son créateur.

\subsection{Droits des utilisateurs}
Lorsqu'un utilisateur s'inscrit dans la base de donnée, on lui attribut un droit 0. Les différents droits sont définis par des entiers:
\begin{itemize}
	\item 0 pour un utilisateur lambda
	\item 1 pour un admistrateur
	\item 2 pour un super-administrateur
\end{itemize}
Un administrateur a la possibilité de révoquer ou d'augmenter les droits d'un utilisateur ayant des droits inférieurs ou égaux à lui-même. Ceci se fait sur la page \texttt{admin.php}, page seulement accessible par les administrateurs. C'est en vérifiant les attributs de la session actuelle \texttt{\$\_SESSION} qu'on détermine si l'utilisateur est bien un admin. Il n'a pas de restriction concernant l'ajout de karas à la queue et la modifier à sa guise. Il peut enfin rajouter des playlists entières à la queue.
